\title{臺北帝國大學文政學部史學科一般報告樣式模板}
\author{南寧笑笑僧\footnote{\ttfamily 臺北帝國大學三民主義硏究所匪情及對匪鬥爭組碩鼠菸酒僧\newline
  \hspace*{4ex}100 臺北市中正區思源街16--2號復明閣; email你懂的,不懂就算了}}

\begin{document}

\maketitle

\section*{緣起}

慶豐四年秋,南寧笑笑僧負笈埋冤。地利人和,百廢待興。乃修模板,依學報舊制,復刻各色樣式於其上。幸甚至哉,特作文以記之。

以上是鬼扯。其實是貧僧常因標點之類蠢問題被釘得雨打沙灘萬點坑。爲了減輕恩師的工作量,以及讓貧僧自己的作業看起來不要那麼噁心,便發奮圖強,依帝大歷史學報撰寫模板(template)。\fcite{}{帝大歷史學報撰稿格式}{。但一般報告自無摘要之類元素。}又由於貧僧一年總有十二三個月不想用Microsoft Word,當然上\TeX(說得好像你會寫Word模板似的)。

\section*{特色}

帝大歷史學報有特出之引用格式:
\begin{quote}
中日韓文專書:作者,《書名》(出版地:出版者,公元年分),頁碼。\\
中日韓文論文:作者,〈篇名〉,《刊物名稱》×卷×期(公元年份,出版地),頁碼。\\
引用原版或影印版古籍,請務必註明版本與卷頁。影印版古籍請註明現代出版項。
\end{quote}
本模板均已考慮,且以近期刊載之論文爲重要參考,處理標點、空格等細部差異。程序可據文獻爲中日韓文或西文適配正確輔助文字及標點符號,排佈時中日韓在前,西文在後,且以筆畫排列。歷史文獻格式規則甚多,如筆畫排列之上,又有朝代及國籍排列,皆西文所無,本模板亦勉力實現。

文獻引用外,本模板亦力圖重現帝大歷史學報樣式,如腳註用楷體等,此不一一。

\section*{入門}

開發環境:Windows 10 Education x64 (201607) + TeXLive 2016\footnote{\ttfamily 帝大有 cw\TeX 發行版,然其文檔竟敢利用小說反黨、利用字體偸婊我皇皇史學科,僅憑這點就該槍斃八十七次。}

報告應在Report.tex撰寫,參考文獻則入ref.bib。如需微調格式,可修NTUHist.tex。NTUBib.tex用於提供基於biblatex之參考文獻樣式(獨立以便其他模板調用),Asian.tex處理東亞與西文之區別,cjkindent.sty爲中文段落宏包,一般用戶均不需理解。

\section*{特有命令}

\verb|\ccite{文獻}| 若腳註之中文文獻不需任何說明,可使用本命令。

\verb|\ecite{文獻}| 若腳註之英文文獻不需任何說明,可使用本命令。

\verb|\fcite{文獻前說明}{文獻}{文獻後說明}| 若腳註中需要說明文字,可使用本命令。

得益於biblatex之驅動,本模板可將文獻分類輸出。目前暫分爲「史料文獻」與「近人硏究」\footnote{\ttfamily 此亦帝大歷史學報分法。}。塡ref.bib時,史料文獻需加形如 \verb|entrysubtype = {明},| 之語句(標記朝代,用於排序及標點,日本整理或撰寫之古籍標 \verb|entrysubtype = {日},| (似朝代而非也),海峽兩岸整理之古籍則加 \verb|keyword = {古籍},|,現代中日韓文需據語言加形如 \verb|language = {chinese},| 之語句(日韓論文還需手動在作者名字前加國籍)。

\section*{文獻分類}

\subsection*{網頁/online}
史學科少用,而三研所較常用。

\subsection*{文章/article}
中文\ccite{明代家庭的權力結構及其成員間的關係}

英文\ccite{WidowsinKinship}

\subsection*{專書/book}
中文\ccite{中國家庭史明清時期123}

英文\ecite{Disgracefulmatters}

\subsection*{章節/inbook}
他編\ccite{清代法上的寡婦和訟師}

自編\ccite{明清貞節的典型}

\subsection*{叢書單本/incollection}
因中文叢書常有子集,如四庫全書有「經」「史」「子」「集」四部,本模板自當處理。寫ref.bib時,booktitle入叢書名,booksubtitle入叢書子集。

舉例\ccite{內則衍義}

\subsection*{學位論文/thesis}
舉例\ccite{由典範到規範}

\subsection*{古籍影本/ancient}
一般biblatex體系所無。舉例\ccite{十七史商榷}

%若不需分類,可衹用\printbibliography。
\section*{引用書目}

\ttfamily\selectfont
\begin{refcontext}[sorting = centy]
\printbibliography[filter = {hsource}, title = {\large 一、史料文獻}]
\end{refcontext}
\begin{refcontext}[sorting = centy]
\printbibliography[filter = {nothsource}, title = {\large 二、近人研究}]
\end{refcontext}

\end{document}